\documentclass{beamer}
% beamer is 128x96 mm
\usepackage{pgf,tikz}
\usepackage{import}
\usepackage{subfig}
\usepackage{float}
\usepackage{xmpmulti}
\usepackage{hyperref}

\usetikzlibrary{arrows}

\title{The Power of Transformations in Geometry}
\author{Ganesh Ajjanagadde \and Shantanu Jain \and James Thomas}
\date{\today}

\begin{document}
\maketitle

%AREA REVIEW SLIDE-------------------------------------------------------------
\begin{frame}
\frametitle{Areas of Shapes}

\begin{columns}[c]

\column{0.5\textwidth}
\begin{figure}[t]
\onslide<1-> \includegraphics[height=0.3\textheight]{./img/rectangle.pdf}
\end{figure}
\begin{figure}[b]
\onslide<3-> \includegraphics[height=0.3\textheight]{./img/right_triangle.pdf}
\end{figure}

\column{0.5 \textwidth}
\onslide<1-> Area =
\onslide<2-> $AB \times BC$ \\
\vspace{40mm}
\onslide<3-> Area =
\onslide<4-> $\frac{AB \times BC}{2}$

\end{columns}
\end{frame}
%AREA REVIEW SLIDE-------------------------------------------------------------

%SIMILAR TRIANGLES REVIEW SLIDE------------------------------------------------
\begin{frame}
\frametitle{Similar Triangles}

\begin{figure}[t]
\begin{center}
\onslide<1-> \includegraphics[height=0.4\textheight]{./img/similar_triangles.pdf}
\end{center}
\end{figure}

\onslide<1-> All equilateral triangles are similar to each other!

\end{frame}
%SIMILAR TRIANGLES REVIEW SLIDE------------------------------------------------

%PYTHAGOREAN THEOREM STATEMENT SLIDE-------------------------------------------
\begin{frame}
\frametitle{The Pythagorean Theorem}

\begin{figure}[t]
\begin{center}
\onslide<1-> \includegraphics[height=0.6\textheight]{./img/pythag_thm/pythag_thm_statement.pdf}
\end{center}
\end{figure}

\onslide<2-> Over 1000 proofs known for this theorem!

\end{frame}
%PYTHAGOREAN THEOREM STATEMENT SLIDE-------------------------------------------

%PYTHAGOREAN THEOREM PROOF-----------------------------------------------------
\begin{frame}
\frametitle{A Proof of the Pythagoren Theorem}
\multiinclude[format=pdf,graphics={height=0.5\textheight},start=1]{./img/pythag_thm/pythag_thm_proof_new}

\pause

\begin{align*}
\onslide<5-> {(a + b)^2 &= c^2 + 4\frac{ab}{2}} \\
\onslide<6-> {a^2 + 2ab + b^2 &= c^2 + 2ab} \\
\onslide<7-> {a^2 + b^2 &= c^2}
\end{align*}
\end{frame}
%PYTHAGOREAN THEOREM PROOF-----------------------------------------------------

%NAPOLEON THEOREM STATEMENT SLIDE----------------------------------------------
\begin{frame}
\frametitle{Napoleon's Theorem}

\begin{columns}[c]
\column{0.5\textwidth}
\onslide<1-> \includegraphics[height=0.6\textheight]{./img/napoleon.jpg}

\column{0.5\textwidth}
\onslide<2-> \includegraphics[height=0.6\textheight]{./img/napoleon_thm/napoleon_thm_proof-start_new.pdf}
\end{columns}
\vspace{10mm}
\onslide<2-> {For an arbitrary $\triangle ABC$, outer triangles $\triangle ABC_1, \triangle BCA_1, \triangle CAB_1$ are equilateral.
Then, the theorem claims that $\triangle DEF$ is equilateral, where $D, E, F$ are the centers of the outer triangles. }
\end{frame}
%NAPOLEON THEOREM STATEMENT SLIDE----------------------------------------------

%NAPOLEON THEOREM ANIMATION SLIDE--------------------------------------------------
\begin{frame}
\frametitle{Two Interesting Rotations}

\begin{columns}[c]

\column{0.5\textwidth}
\onslide<1-> \includegraphics[height=0.8\textheight]{./img/napoleon_thm/napoleon_thm_proof-start_new.pdf}

\pause
\column{0.5\textwidth}
\multiinclude[format=pdf,graphics={height=0.8\textheight},start=1]{./img/napoleon_thm/napoleon_thm_proof_new}

\end{columns}
\end{frame}
%NAPOLEON THEOREM ANIMATION SLIDE--------------------------------------------------

%NAPOLEON THEOREM PROOF SLIDE------------------------------------------------------
\begin{frame}[fragile]
\frametitle{A Proof of Napoleon's Theorem}

\begin{columns}[c]

\column{0.5\textwidth}
\multiinclude[format=pdf,graphics={height=0.8\textheight},start=1]{./img/napoleon_thm/napoleon_thm_proof_diag}
%\onslide<1-> \includegraphics[width=0.9\textwidth]{./img/napoleon_thm/napoleon_thm_proof_new-14.pdf}

\column{0.5\textwidth}
\begin{small}
\begin{itemize}
\onslide<1-> \item  $\frac{AF}{AB} = \frac{AE}{AB_1}$
\onslide<2-> \item  $AF = AG, AE = AH$
\onslide<3-> \item  $\frac{AG}{AB} = \frac{AH}{AB_1}$
\onslide<4-> \item  $\angle GAH = \angle BAB_1$ $\Rightarrow \triangle AGH$ is similar to $\triangle ABB_1$, with ratio $\frac{AG}{AB}$
\onslide<5-> \item  $\frac{FE}{BB_1} = \frac{GH}{BB_1} = \frac{AG}{AB}$
\onslide<6-> \item  $\frac{DE}{BB_1} = \frac{IJ}{BB_1} = \frac{CI}{CB}$
\onslide<7-> \item $\frac{AG}{AB} =  \frac{AF}{AB} $ \onslide<8-> $= \frac{CD}{CB} = \frac{CI}{CB}$
\onslide<9-> \item  $\frac{FE}{BB_1} = \frac{DE}{BB_1} \Rightarrow FE = DE$
\onslide<10-> \item  $\triangle DEF$ is equilateral
\end{itemize}
\end{small}
\end{columns}
\end{frame}
%NAPOLEON THEOREM PROOF SLIDE--------------------------------------------------

%MORLEY THEOREM SLIDE----------------------------------------------------------
\begin{frame}
\frametitle{More Fun Geometry}

\begin{itemize}
\item If you are interested, check out \url{http://www.cs.toronto.edu/~mackay/conway.pdf}.
\item It contains a proof of the very remarkable ``Morley Theorem''.
\end{itemize}

\end{frame}
%MORLEY THEOREM SLIDE----------------------------------------------------------

\end{document}
